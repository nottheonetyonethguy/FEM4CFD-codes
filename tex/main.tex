\documentclass[12pt, oneside]{article} 
\usepackage{amsmath, amsthm, amssymb, calrsfs, wasysym, verbatim, bbm, color, graphics, geometry}

\geometry{tmargin=.75in, bmargin=.75in, lmargin=.75in, rmargin = .75in}  

\newcommand{\R}{\mathbb{R}}
\newcommand{\C}{\mathbb{C}}
\newcommand{\Z}{\mathbb{Z}}
\newcommand{\N}{\mathbb{N}}
\newcommand{\Q}{\mathbb{Q}}
\newcommand{\Cdot}{\boldsymbol{\cdot}}

\newtheorem{thm}{Theorem}
\newtheorem{defn}{Definition}
\newtheorem{conv}{Convention}
\newtheorem{rem}{Remark}
\newtheorem{lem}{Lemma}
\newtheorem{cor}{Corollary}


\title{\textbf{FEM4CFD Notes}}
\author{Bibek Yonzan}
\date{2025}

\begin{document}

\maketitle
\tableofcontents

\vspace{.25in}

\section{1D Advection Diffusion Equation}

\subsection{Problem Statement}

The advection diffusion reaction equation models the transport phenomena involving advection, diffusion, and reaction. Advection can be defined by the transport of the substance by bulk fluid motion, whereas diffusion can be defined by the spreading of the substance due to random molecular motion. Similarly, reaction is defined by the creation, destruction, or transformation of the substance. The advection diffusion reaction equation is used to various phenomena such as heat transfer in flowing fluids, or even chemical reactions in flowing media. This section is concerned with the steady state solution of the advection diffusion reaction equation. \\
The advection diffusion equation is of the type
   \begin{equation}
       \frac{\partial\Phi}{\partial t} + \frac{\partial{\boldsymbol{F_i}}}{\partial{x_i}} + \frac{\partial\boldsymbol{G_i}}{\partial{x_i}} + \boldsymbol{Q} = 0
   \end{equation}
where in general, $\Phi$ is a basic dependent, vector variable, \textbf{Q} is a source or reaction term and flux matrices \text{F} and \textbf{G} are such that
    \begin{equation}\label{adr_ex}
          \begin{aligned}
	    \boldsymbol{F_i} = \boldsymbol{F_i}(\boldsymbol{\Phi})\\
	    \boldsymbol{G_i} = \boldsymbol{G_i}\frac{\partial{\boldsymbol{\Phi}}}{\partial{x_j}}\\
	    \boldsymbol{Q} = \boldsymbol{Q}(x_i, \boldsymbol{\Phi})
          \end{aligned}
    \end{equation}
    A linear reaction between the source term and the scalar variable is referred to as the reaction term. In (\ref{adr_ex}), $x_i$ and $i$ refer to Cartesian coordinates and the associated quantities and as a whole \ref{adr_ex} can be termed the transport eqation with $\boldsymbol{F}$ and $\boldsymbol{G}$ as the advection and diffusion coefficients, respectively. In scalar terms \ref{adr_ex} becomes,
	\begin{equation}
	   \begin{aligned}
		\Phi \rightarrow \phi, \quad \quad \quad \quad \boldsymbol{Q} \rightarrow Q(x_i, \phi) = c \phi\\
		\boldsymbol{F_i} \rightarrow F_i = U_i\phi, \quad \quad \quad \quad \boldsymbol{G_i} \rightarrow G_i = -k \frac{\partial\phi}{\partial t}
	    \end{aligned}
	\end{equation}
Thus, the equation becomes
   \begin{equation}
\frac{\partial\phi}{\partial t} + \frac{\partial{(U_i\phi)}}{\partial{x_i}} - \frac{\partial}{\partial{x_i}}(k \frac{\partial\phi}{\partial{x_i}}) + Q = 0
   \end{equation}
Here, \textit{U} is the velocity field and $\phi$ is the scalar quantity being transported by this velocity. But, diffusion can also occur, and \textit{k} is the diffusion coefficient.\\
A linear reaction term can be written associated, where \textit{c} is a scalar parameter.
\[
    Q = c\ \phi
\]

\end{document}

